\documentclass[useAMS,usenatbib,referee]{biom}
%\documentclass[useAMS,usenatbib,referee]{biom}
%
%
%  Papers submitted to Biometrics should ALWAYS be prepared
%  using the referee option!!!!
%
%
% If your system does not have the AMS fonts version 2.0 installed, then
% remove the useAMS option.
%
% useAMS allows you to obtain upright Greek characters.
% e.g. \umu, \upi etc.  See the section on "Upright Greek characters" in
% this guide for further information.
%
% If you are using AMS 2.0 fonts, bold math letters/symbols are available
% at a larger range of sizes for NFSS release 1 and 2 (using \boldmath or
% preferably \bmath).
%
% The usenatbib command allows the use of Patrick Daly's natbib package for
% cross-referencing.
%
% If you wish to typeset the paper in Times font (if you do not have the
% PostScript Type 1 Computer Modern fonts you will need to do this to get
% smoother fonts in a PDF file) then uncomment the next line
% \usepackage{Times}
%%%%% AUTHORS - PLACE YOUR OWN MACROS HERE %%%%%

\usepackage[figuresright]{rotating}
\usepackage{tikz}
\usepackage{amsmath}
\usepackage[hyphens]{url} % not crucial - just used below for the URL
\usepackage{hyperref}
\usepackage[utf8]{inputenc}
\usepackage{graphicx}
\usepackage{longtable}
\usepackage{booktabs}
%% \raggedbottom % To avoid glue in typesetteing, sbs>>

% Pandoc syntax highlighting
\usepackage{color}
\usepackage{fancyvrb}
\newcommand{\VerbBar}{|}
\newcommand{\VERB}{\Verb[commandchars=\\\{\}]}
\DefineVerbatimEnvironment{Highlighting}{Verbatim}{commandchars=\\\{\}}
% Add ',fontsize=\small' for more characters per line
\usepackage{framed}
\definecolor{shadecolor}{RGB}{248,248,248}
\newenvironment{Shaded}{\begin{snugshade}}{\end{snugshade}}
\newcommand{\AlertTok}[1]{\textcolor[rgb]{0.94,0.16,0.16}{#1}}
\newcommand{\AnnotationTok}[1]{\textcolor[rgb]{0.56,0.35,0.01}{\textbf{\textit{#1}}}}
\newcommand{\AttributeTok}[1]{\textcolor[rgb]{0.77,0.63,0.00}{#1}}
\newcommand{\BaseNTok}[1]{\textcolor[rgb]{0.00,0.00,0.81}{#1}}
\newcommand{\BuiltInTok}[1]{#1}
\newcommand{\CharTok}[1]{\textcolor[rgb]{0.31,0.60,0.02}{#1}}
\newcommand{\CommentTok}[1]{\textcolor[rgb]{0.56,0.35,0.01}{\textit{#1}}}
\newcommand{\CommentVarTok}[1]{\textcolor[rgb]{0.56,0.35,0.01}{\textbf{\textit{#1}}}}
\newcommand{\ConstantTok}[1]{\textcolor[rgb]{0.00,0.00,0.00}{#1}}
\newcommand{\ControlFlowTok}[1]{\textcolor[rgb]{0.13,0.29,0.53}{\textbf{#1}}}
\newcommand{\DataTypeTok}[1]{\textcolor[rgb]{0.13,0.29,0.53}{#1}}
\newcommand{\DecValTok}[1]{\textcolor[rgb]{0.00,0.00,0.81}{#1}}
\newcommand{\DocumentationTok}[1]{\textcolor[rgb]{0.56,0.35,0.01}{\textbf{\textit{#1}}}}
\newcommand{\ErrorTok}[1]{\textcolor[rgb]{0.64,0.00,0.00}{\textbf{#1}}}
\newcommand{\ExtensionTok}[1]{#1}
\newcommand{\FloatTok}[1]{\textcolor[rgb]{0.00,0.00,0.81}{#1}}
\newcommand{\FunctionTok}[1]{\textcolor[rgb]{0.00,0.00,0.00}{#1}}
\newcommand{\ImportTok}[1]{#1}
\newcommand{\InformationTok}[1]{\textcolor[rgb]{0.56,0.35,0.01}{\textbf{\textit{#1}}}}
\newcommand{\KeywordTok}[1]{\textcolor[rgb]{0.13,0.29,0.53}{\textbf{#1}}}
\newcommand{\NormalTok}[1]{#1}
\newcommand{\OperatorTok}[1]{\textcolor[rgb]{0.81,0.36,0.00}{\textbf{#1}}}
\newcommand{\OtherTok}[1]{\textcolor[rgb]{0.56,0.35,0.01}{#1}}
\newcommand{\PreprocessorTok}[1]{\textcolor[rgb]{0.56,0.35,0.01}{\textit{#1}}}
\newcommand{\RegionMarkerTok}[1]{#1}
\newcommand{\SpecialCharTok}[1]{\textcolor[rgb]{0.00,0.00,0.00}{#1}}
\newcommand{\SpecialStringTok}[1]{\textcolor[rgb]{0.31,0.60,0.02}{#1}}
\newcommand{\StringTok}[1]{\textcolor[rgb]{0.31,0.60,0.02}{#1}}
\newcommand{\VariableTok}[1]{\textcolor[rgb]{0.00,0.00,0.00}{#1}}
\newcommand{\VerbatimStringTok}[1]{\textcolor[rgb]{0.31,0.60,0.02}{#1}}
\newcommand{\WarningTok}[1]{\textcolor[rgb]{0.56,0.35,0.01}{\textbf{\textit{#1}}}}

% tightlist command for lists without linebreak
\providecommand{\tightlist}{%
  \setlength{\itemsep}{0pt}\setlength{\parskip}{0pt}}



%%%%%%%%%%%%%%%%%%%%%%%%%%%%%%%%%%%%%%%%%%%%%%%%

\setcounter{footnote}{2}

\title[]{Average treatment effect of cholesterol-lowering medication and
average systolic blood pressure (SBP), mm Hg.}

\author{ Jackson
Gazin \email{\href{mailto:gazij22@wfu.edu}{\nolinkurl{gazij22@wfu.edu}}} \\ Department
of Statistics, Wake Forest University  \and
		 Ashley
Mullan \email{\href{mailto:mullae22@wfu.edu}{\nolinkurl{mullae22@wfu.edu}}} \\ Department
of Statistics, Wake Forest University  \and
		 Anh
Nguyen \email{\href{mailto:nguyp22@wfu.edu}{\nolinkurl{nguyp22@wfu.edu}}} \\ Department
of Statistics, Wake Forest University 
	   }


\begin{document}


\date{{\it Received Dec} 2023}

\pagerange{\pageref{firstpage}--\pageref{lastpage}} \pubyear{2023}

\volume{0}
\artmonth{January}
\doi{0000-0000-0000}

%  This label and the label ``lastpage'' are used by the \pagerange
%  command above to give the page range for the article

\label{firstpage}

%  pub the summary here

\begin{abstract}
We investigate the average treatment effect among the treated (ATT) of
cholesterol-lowering medication on the mean systolic blood pressure (mm
Hg). Using data from the National Health and Nutrition Examination
Survey (NHANES), we fit a propensity score model to estimate the ATT
among adults living in the United States of America.
\end{abstract}

%
%  Please place your key words in alphabetical order, separated
%  by semicolons, with the first letter of the first word capitalized,
%  and a period at the end of the list.
%

\begin{keywords}
cholesterol-lowering medicationsystolic blood pressureblood pressure
control.
\end{keywords}

\maketitle

\hypertarget{intro}{%
\section{Introduction}\label{intro}}

Controlling blood pressure (BP) reduces the risk for cardiovascular
disease. However, the prevalence of BP control (i.e., systolic BP
\textless{} 140 and diastolic BP \textless{} 90) among US adults with
hypertension has decreased since 2013. We invite teams to analyze
publicly available data from US adults to help identify potential causes
or correlates of worsening BP control among US adults with hypertension
over the past decade, as this may allow for development of effective
interventions to help control BP and prevent cardiovascular disease.

\hypertarget{methods}{%
\section{Materials and methods}\label{methods}}

\hypertarget{data}{%
\subsection{Data}\label{data}}

The National Health and Nutrition Examination Survey (NHANES) combines
interviews and physical examinations to assess the health and
nutritional status of adults and children in the United States of
America. The program started in the early 1960s and has been conducted
every two years since 1999. The survey samples from a nationally
representative 5,000 persons each year. The participants are located in
counties across the country, 15 of which are visited each year. The
interview asks questions about demographic, socioeconomic, dietary, and
health-related questions. The examination consists of medical, dental,
physiological measurements, and laboratory tests.

The NHANES dataset we are using can be downloaded from \textbf{cite} .
The dataset contains information from the survey from 1999 to 2020 with
a sample of 59,799 rows and 111 chosen columns. Each row is a
noninstitutionalized US adults participated in the survey between 1999
and 2020. The columns contain information about demographics, blood
pressure levels, hypertension status, antihypertensive medication usage,
and co-morbidities.

For this analysis, we had 38977 rows with NA values. We decided to deal
with this by removing all the na values. We decided this was preferable
to removing certain columns since we were still left with 20,822 data
points which is still an extremely large data set.

\hypertarget{statistical-methods}{%
\subsection{Statistical methods}\label{statistical-methods}}

We fit a propensity score model withTotal cholesterol, mg/dL as our
explanatory variable since this was the only variable and our taking
cholesterol medication as our response variable since this is our
exposure variable. We fit our propensity score model with a Logistic
Regression model. We then fit our outcome model to estimate the average
treatment effect among the treated with a linear regression model with
cholesterol medication as our explanatory variable and Systolic blood
pressure (SBP), mm Hg as our response variable. We use a weighted
mirrored histogram, ECDF plots, and Love Plots to check our the
appropriateness of our propensity score model to proceed with our final
model.

\hypertarget{exploratory-data-analysis}{%
\subsubsection{Exploratory data
analysis}\label{exploratory-data-analysis}}

\hypertarget{modeling}{%
\subsubsection{Modeling}\label{modeling}}

\hypertarget{results-results}{%
\section{Results \{\$results\}}\label{results-results}}

\hypertarget{study-population}{%
\subsection{Study population}\label{study-population}}

We aim answer our causal question by fitting an average treatment effect
among the treated. Our causal question is as follows: Among those who
take cholesterol lowering medication, does taking this cholesterol
lowering medication change their systolic blood pressure?

\hypertarget{propensity-score-model-and-diagnostics}{%
\subsection{Propensity score model and
Diagnostics}\label{propensity-score-model-and-diagnostics}}

We fit a propensity score model with total cholesterol, mg/dL as our
explanatory variable since this was the only variable and our taking
cholesterol medication as our response variable since this is our
exposure variable. We fit our propensity score model with a Logistic
Regression model. We then examined a Mirrored Histogram of our
propensity scores across both exposure groups. As we can see in the
table below, we have significant overlap for the propensity scores
across both groups. This suggests very little evidence of having a
positivity violation in our model which is great.

We then used our propensity score model to generate our average
treatment affect among the treated weights. We generated these weights
since they matched our causal question.

\hypertarget{average-treatment-effect-among-the-treated}{%
\subsection{Average treatment effect among the
treated}\label{average-treatment-effect-among-the-treated}}

\hypertarget{discussion}{%
\section{Discussion}\label{discussion}}

\hypertarget{supplement}{%
\section{Supplementary information}\label{supplement}}

The data can be downloaded from GitHub or accessed via the
cardioStatsUSA R package. For both the file and information about the R
package, see \url{https://github.com/jhs-hwg/cardioStatsUSA}.

All code for the analysis can be accessed at \textbf{link github}

\hypertarget{acknowledge}{%
\section{Acknowledgement}\label{acknowledge}}

\hypertarget{sec:1}{%
\section{Section title}\label{sec:1}}

Text with citations by \citet{heagerty2000time},
\citep{pepe2003statistical}.

\hypertarget{sec:2}{%
\subsection{Subsection title}\label{sec:2}}

as required \citep{hoerl1970ridge, zou2005regularization}. Don't forget
to give each section and subsection a unique label (see Sect.
\ref{sec:1}).

\hypertarget{paragraph-headings}{%
\paragraph{Paragraph headings}\label{paragraph-headings}}

Use paragraph headings as needed.

\hypertarget{equations}{%
\subsection{Equations}\label{equations}}

Here is an equation:

\[ f_{X}(x) = \left(\frac{\alpha}{\beta}\right)\left(\frac{x}{\beta}\right)^{\alpha-1}e^{-\left(\frac{x}{\beta}\right)^{\alpha}}; \alpha,\beta,x > 0 \]

Here is another: \begin{align}
a^2+b^2=c^2
\end{align}

Inline equations: \(\sum_{i = 2}^\infty\{\alpha_i^\beta\}\)

\hypertarget{figures-and-tables}{%
\section{Figures and tables}\label{figures-and-tables}}

\hypertarget{figures-coming-from-r}{%
\subsection{Figures coming from R}\label{figures-coming-from-r}}

\hypertarget{normal-figure-embedded-in-text}{%
\paragraph{Normal figure embedded in
text}\label{normal-figure-embedded-in-text}}

\begin{verbatim}
## Warning in plot.formula(runif(25) ~ runif(25)): the formula 'runif(25) ~
## runif(25)' is treated as 'runif(25) ~ 1'
\end{verbatim}

\begin{figure}
\centering
\includegraphics{final-project_files/figure-latex/fig2-1.pdf}
\caption{Output from \texttt{pdf()}}
\end{figure}

\clearpage

\hypertarget{tables-coming-from-r}{%
\subsection{Tables coming from R}\label{tables-coming-from-r}}

\begin{Shaded}
\begin{Highlighting}[]
\FunctionTok{print}\NormalTok{(xtable}\SpecialCharTok{::}\FunctionTok{xtable}\NormalTok{(}\FunctionTok{head}\NormalTok{(mtcars)[,}\DecValTok{1}\SpecialCharTok{:}\DecValTok{4}\NormalTok{], }
\AttributeTok{caption =} \StringTok{"Caption centered under table"}\NormalTok{, }\AttributeTok{label =} \StringTok{"tab1"}\NormalTok{), }
\AttributeTok{comment =} \ConstantTok{FALSE}\NormalTok{, }\AttributeTok{timestamp =} \ConstantTok{FALSE}\NormalTok{, }\AttributeTok{caption.placement =} \StringTok{"top"}\NormalTok{)}
\end{Highlighting}
\end{Shaded}

\begin{table}[ht]
\centering
\caption{Caption centered under table} 
\label{tab1}
\begin{tabular}{rrrrr}
  \hline
 & mpg & cyl & disp & hp \\ 
  \hline
Mazda RX4 & 21.00 & 6.00 & 160.00 & 110.00 \\ 
  Mazda RX4 Wag & 21.00 & 6.00 & 160.00 & 110.00 \\ 
  Datsun 710 & 22.80 & 4.00 & 108.00 & 93.00 \\ 
  Hornet 4 Drive & 21.40 & 6.00 & 258.00 & 110.00 \\ 
  Hornet Sportabout & 18.70 & 8.00 & 360.00 & 175.00 \\ 
  Valiant & 18.10 & 6.00 & 225.00 & 105.00 \\ 
   \hline
\end{tabular}
\end{table}

Table \ref{tab1} shows these numbers. Some of those numbers are plotted
in Figure \ref{fig:fig1}.

\begin{Shaded}
\begin{Highlighting}[]
\FunctionTok{head}\NormalTok{(mtcars[,}\DecValTok{1}\SpecialCharTok{:}\DecValTok{4}\NormalTok{])}
\end{Highlighting}
\end{Shaded}

\begin{verbatim}
##                    mpg cyl disp  hp
## Mazda RX4         21.0   6  160 110
## Mazda RX4 Wag     21.0   6  160 110
## Datsun 710        22.8   4  108  93
## Hornet 4 Drive    21.4   6  258 110
## Hornet Sportabout 18.7   8  360 175
## Valiant           18.1   6  225 105
\end{verbatim}


\bibliographystyle{biom}
\bibliography{bibliography.bib}


\label{lastpage}


\end{document}
